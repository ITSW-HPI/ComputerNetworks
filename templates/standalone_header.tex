\usepackage[utf8]{inputenc}
\usepackage[T1]{fontenc}


\RequirePackage{xcolor}

%% HPI color definitions according to the design manual
% These do not exactly match the RGB values used in the Powerpoint slide master due to unknown reasons
\definecolor{hpiyellow}{RGB}{246,168,0}
\definecolor{hpiorange}{RGB}{221,97,8}
\definecolor{hpired}{RGB}{177,6,58}
\definecolor{hpigray}{RGB}{90,96,101}
\definecolor{hpiblue}{RGB}{0,122,158}


\renewcommand{\sfdefault}{neosans}
% Different font weights for neosans
\newcommand{\textl}[1]{{\fontseries{l}\selectfont #1}} % light
\newcommand{\textm}[1]{{\fontseries{m}\selectfont #1}} % medium, same as default weight
\newcommand{\textsb}[1]{{\fontseries{sb}\selectfont #1}} % semibold
\newcommand{\textmb}[1]{{\fontseries{mb}\selectfont #1}} % bold, same as \textbf
\newcommand{\texteb}[1]{{\fontseries{eb}\selectfont #1}} % extra bold
\newcommand{\textub}[1]{{\fontseries{ub}\selectfont #1}} % ultra bold

\tikzset{every picture/.style={/utils/exec={\sffamily}}}
\tikzset{flipflop RSflanke/.style={
  flipflop,
  flipflop def={t1=S, t2=C, c2=1, t3=R, t6=Q, t4={\ctikztextnot{Q}}}
}}

\usetikzlibrary{calc}
\usetikzlibrary{positioning}
