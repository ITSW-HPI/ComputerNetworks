% -*-LaTeX-*-

\newglossaryentry{signal}{name=signal, description={A (usually deliberately caused) change of a physical quantity; this change propagates in space and time and can be measured.}}

\newglossaryentry{simplex}{name=simplex, description={Using a physical medium or a channel only in single direction; the roles of transmitter and receiver stay fixed.}}

\newglossaryentry{half-duplex}{name=half-duplex, description={Using a physical medium or a channel alternatingly, with at most one participant, at any time, transmitting to the other participant.}}

\newglossaryentry{full-duplex}{name=full-duplex, description={Allowing both participants to send data simultaneously; can be achieved by using multiple simplex channels or exploiting buffers and differences in data rates.}}

\newglossaryentry{duplexing}{name=duplexing, description={How to organise the exchange of data between \emph{two} entities \emph{with each other}. }}

\newglossaryentry{interference}{name=interference, description={Two or more \emph{signals} arriving at a receiver in the same channel (same physical medium, same location, same frequency band, same time, \dots); it is \emph{not} caused by noise.}}


\newglossaryentry{fdd}{name={Frequency-division duplexing (FDD)}, description={Frequency-division duplexing divides a physical medium into two channels by assigning separate frequency bands to each transmitter in a communicating pair. }}

\newglossaryentry{tdd}{name={Time-division duplexing (TDD)}, description={Time-division duplexing divides the use of a physical medium into discrete time slots, alternating per direction, to organize the communication between two entities in a communication pair.  }}

\newglossaryentry{channel}{name=channel, description={A channel describes a collection of resources necessary for communication, in either a physical or logical sense. Physical channels (usually) comprise location, physical medium, time, frequency band, and possibly code, along with characteristic changes made to a signal over such a physical channel (e.g., frequency-dependent attenuation). Logical channels can take on a wide range of characteristics, up to very abstract notions of the channel between, e.g., two humans. }}


\newglossaryentry{multiplexing}{name=multiplexing, description={How to share a medium/channel between different \emph{pairs} of communication partners. In general it is allowed for one entity to participate in multiple pairs (e.g., a wireless access points forms pairs with each device in its wireless cell).}}

\newglossaryentry{tdm}{name={Time-division multiplexing (TDM)}, description={Time-division multiplexing shares a resource in time, splitting it up in time slots. These time slots are often, but not necessarily, of constant length. Usually, but not always, some guard times are necessary between them.}}

\newglossaryentry{fdm}{name={Frequency-division multiplexing (FDM)}, description={Frequency-division multiplexing shares a resource by dividing it in separate frequency bands. Usually, but not always, some guard bands are necessary between these useful bands, creating overhead. Frequency bands are usually, but not necessarily, of constant width.}}

\newglossaryentry{circuit}{name=circuit, description={A continuous physical medium along which a signal can propagate from a transmitter to a receiver, created by configuring physical components in intermediate devices. Circuits typically need to be established before first use and can be destroyed after end of use.}}

\newglossaryentry{circuitswitching}{name=circuit switching, description={
 Organizing data exchange in a network via circuits, along with their creation and tearing down. 
}}

\newglossaryentry{packetswitching}{name=Packet switching, description={Organizing data exchange dividing data transmission into packets composed of payload and meta-data, using physical channels only between adjacent devices, and forwarding packets.} }

\newglossaryentry{bursty}{name=bursty traffic, description=Traffic with a high ratio of peak to average required data rate.}

\newglossaryentry{forwarding}{name=forwarding, description={In a packet-switched network, the act of receiving a packet, determining its next hop, preparing it for transmission, and transmitting it to the next hop. Forwarding is a data-plane activity.   Forwarding typically, but not necessarily, uses information prepared by control-plane protocols (e.g. routing tables).}  }


\newglossaryentry{routing}{name=routing, description={The process of filling routing tables.}}

\newglossaryentry{storeandforward}{name=store-and-forward, description={A variant of a packet-switched network where a packet first has to be received completely before its meta-data is analyzed, the outgoing interface is determined, the packet is moved to the outoing interface and eventually transmitted. Cut-through networks are the alternative.}}

\newglossaryentry{metadata}{name={meta data}, description={Data that describes how to process a packet. Examples include source and destination address, sequence numbers, checksums, etc. Usually found in a header or trailer of a packet, sometimes implicitly represented by, e.g., a position of time slot within a periodic time structure.}}

\newglossaryentry{dataplane}{name={data plane}, description={Concept to collect all functionality concerned with transporting data per se; not concerned with decisions}}

\newglossaryentry{controlplane}{name={control place}, description={Concept to collect all functionality concerned with taking decisions to control data-plane functionality.}}

\newglossaryentry{mac}{name={Medium Access Control (MAC)}, description={A control function that decides which entity is allowed to transmit, at which point in time, using which multiplexing option.}}


\newglossaryentry{protocol}{name=protocol, description={A set of rules and data formats that mandate how two peers communicate with each other. Behavior of a peer is often described using a finite state machine.}}

\newglossaryentry{path}{name=path, description={A sequence of network entities (e.g., end devices, switches, routers, \dots) that a packet traverses on its way from source to destination.}}

\newglossaryentry{msc}{name={Message Sequence Chart (MSC)}, description={A message sequence chart shows the transmission of data (via circuits, packets, \dots) by drawing the time axis vertically, facing downwards, and a distance axis horizontally. Propation speed is then shown by the angle of signals.}}

\newglossaryentry{rfc}{name={Request for Comment (RFC)}, description={A document of the IETF}}

\newglossaryentry{payload}{name=payload, description={Data that is handed down from a higher layer to a service access point for transmission / is handed upwards from an SAP to a higher layer. Payload is a relative term between layers; it can encompass headers of higher layers. Sometimes, \emph{user payload} is used to designate the actual, ultimate data to be transmitted without any network headers of any layers being added. }}

\newglossaryentry{encapsulating}{name=encapsulating, description={Adding meta data at the front or at the back of a payload message.}}

\newglossaryentry{header}{name=header, description={Meta data added at the start of message (to be transmitted first).}}

\newglossaryentry{trailer}{name=trailer, description={Meta data added at the end of message (to be transmitted last).}}

\newglossaryentry{routingtable}{name={routing table}, description={A table for a node in a network that, conceptually, lists costs the costs to go any destination in the network via any of the immediate neighbors of that node.}}

\newglossaryentry{forwardingtable}{name={forwarding table}, description={A table for a node that lists, for any destination in the network, the next hop to use for the best known path towards that destination. Next hop might include data like interface or name of the neighboring node. Does usually not include cost; usually constructed from routing tables.}}

\newglossaryentry{FEC}{name={Forward Error Correction}, description={Adding redundancy to a data transmission (e.g., to a packet) to combat imperfections of the transmission process. Allows to correct incorrect bits.}}


\newglossaryentry{utp}{name=Unshielded Twisted Pair, description=Unshielded Twisted Pair}

\newglossaryentry{signalbw}{name=signal bandwidth , description={The \textbf{bandwidth} of a \textbf{signal} $g$ is the difference between its upper and lower frequencies in the Fourier series of $g$; for infinite Fourier series, the upper frequency results from cutting off high terms. }}

\newglossaryentry{attenuation}{name=attenuation, description={The ratio of a signal's transmitted power to its received power. Also used for the harmonics of a signal in the same sense.}}

\newglossaryentry{channelbw}{name=channel bandwidth, description={An interval of frequency in which a channel's attenuation is below an acceptable threshold; typically the difference between an upper and a lower cutoff frequency. Usually considered as the \emph{useable frequency range} of a channel.}}

\newglossaryentry{layering}{name=layering, description={The approach of structuring a system by using an abstract modularization where there is a strict order between modules and modules are only allowed to interact with their adjacent modules. }}

\newglossaryentry{serviceprimitive}{name=service primitive, description={The set of operations available at a service interface along with permissible sequences and parameters. }}

\newglossaryentry{frequencydomain}{name=frequency-domain representation, description=A function represented as a sum of sine and cosine functions whose frequencies are integer multiples of the function's base frequency.}

\newglossaryentry{fourieranalysis}{name=Fourier analysis, description={Determining the constants $c$, $a_n$, $b_n$ in a function's Fourier series.}}

\newglossaryentry{DC}{name=direct current (DC), description={The term $1/2 c$ in a function's Fourier series.}}

\newglossaryentry{harmonics}{name=harmonics, description={The terms $a_n$, $b_n$ of a function's Fourier series are its $n$th \textbf{harmonics}. It contains  \textbf{power} proportional to  $a_n^2 + b_n^2$}.}

\newglossaryentry{awgn}{name=Additive White Gaussian Noise (AWGN), description={A model for noise inside a receiver, characterized by being added to the received signal and having the same power in all harmonics. This power follows a Gaussian distribution and is uncorrelated in time.}}

\newglossaryentry{channelcapacity}{name=Channel capacity, description={The capacity of a channel (in the Shannon sense) is an upper bound on the data rate that can be transported over that channel without errors \textbf{after} possible error correction. It is not possible to actually achieve that rate, but it can be approached closely by using error correction over longer and longer codewords, trading off data rate against delay.}}

\newglossaryentry{snr}{name=Signal-to-Noise Ratio (SNR), description={The ratio between the \emph{received} signal power and the noise power at a receiver, usually used in an average-over-time sense (sometimes, also as \emph{instantaneous} SNR).}}

\newglossaryentry{baseband}{name=baseband transmission, description={In baseband transmission, symbol values are directly mapped to a parameter of the signal's physical quantity.}}

\newglossaryentry{broadband}{name=broadband transmission, description={In broadband transmission, symbol values are mapped to a parameter of a \emph{function} of the signal's physical quantity (usually, a periodic function, e.g., a sine function). This function is often called the \emph{carrier}. The mapping process is called \emph{modulation}}}

\newglossaryentry{carrier}{name=carrier, description={In broadband transmission, the \emph{carrier} is the unmodified, usually periodic function used for the signal. It usual has at least one parameter that is modified by modulation of symbol values.}}

\newglossaryentry{modulation}{name=modulation, description={Modulation describes how a symbol value is used to choose a parameter of a signal's function.}}

\newglossaryentry{waveform}{name=waveform, description={The actual function that is sent as a signal after the carrier has been modulated for a particular symbol value. Usually, there is one waveform for every possible symbol.}}

\newglossaryentry{connection}{name=connection, description={State shared between sender and receiver. Needs to be established by \emph{connection setup}.}}

\newglossaryentry{flowcontrol}{name=flow control, description={Regulating the transmission rate of a sender in order not to overwhelm its receiver. \emph{Not} related to the network between them.}}

\newglossaryentry{hammingdistance}{name=Hamming distance, description={Between two bit sequences, the number of 1s in their XOR. For a set of bit sequences, the smallest distance between any two (non-identical) of them. }}

\newglossaryentry{generator}{name=generator, description={A generator matrix for a forward error correction scheme is a matrix $G$ that produces a codeword by mutliplication (from right) with payload. }}

\newglossaryentry{codeword}{name=codeword, description={Payload plus redudancy}}

\newglossaryentry{codinggain}{name={coding gain}, description={The reduction in SNR that can be tolerated when using a forward error correction code without compromising error rate. It is usually expressed as the ratio between (required) SNR for uncoded to coded scheme, and usually written in decibels. It is justifiable to talk about \textbf{the} coding gain of a particular FEC scheme (without referring to a particular error rate) when focusing on SNR regions where the ratio between coded and uncoded schemes becomes constant. This \emph{is} a simplification. }}

\newglossaryentry{arq}{name=Automatic Repeat Request (ARQ), description={A family of backward-recovery protocols based on acknowledgements and timeouts.}}

\newglossaryentry{lan}{name=Local Area Network (LAN), description={A network that spans a range smaller than a regional network but bigger than a personal network. It usually connects more devices or carries more load than would be feasible using a single shared medium. It can be geared to both private and public use and use different technologies (e.g., wired or wireless), often interconnecting between them (e.g., Ethernet and WiFi). It can use multiple address formats. It is usually under single administrative control. }}

\newglossaryentry{vlan}{name=Virtual Local Area Network (VLAN), description={A VLAN creates the illusion of a LAN on top of some other network. It is not necessarily local in a geographic sense; it usually serves to limit broadcast domains. (Note: in a strict operating-system-terminology sense, it would not be a \emph{virtual}) LAN but a \emph{logial} LAN, but that is not commonly used in networking.) }}

\newglossaryentry{repeater}{name=repeater, description={A repeater is a physical-layer device. It accepts a signal and sends out a signal that is in some sense an improved version of that signal (e.g., amplified). Usually, it is not possible to create a perfect copy; usually, at least noise is amplified as well during the process.}}

\newglossaryentry{hub}{name=hub, description={A physical-layer device that assist in plugging together multiple devices into a single collision domain. Hubs exist in passive versions (purely a mechanical device) or combined with repeaters. }}

\newglossaryentry{switch}{name=switch, description={A switch in the general sense is a device that connects multiple links, for e.g. packet or circuit-switching. In the narrower sense, switch designates a link-layer device that connects multiple devices without forming a physical collision domain but does forward broadcast packets to all its ports. Many variations exist! }}


\newglossaryentry{flooding}{name=flooding, description={Forwarding a packet to all reachable devices. Flooding and broadcasting are often used interchangeably. }}

\newglossaryentry{router}{name=router, description={A device that runs routing protocols in its control plane to fill its routing tables, derives forwarding tables from them, and uses them to forward (e.g.) packets. Routers often, but not necessarily, use some form of structured addresses, wheres switches often, but not necessarily, work on flat address spaces. In the narrower sense, routers are devices on the network layer of a typical protocol stack, but that definition has many exceptions. }}


\newglossaryentry{congestion}{name=congestion, description={A link is congested if the sum of the incoming rates exceeds its service rate for a substantial time, leading to queue buildup and eventual packet drops.}}

\newglossaryentry{congestioncontrol}{name=congestion control, description={A (set of) mechanism(s) to ensure that packets are not dropped due to full buffers inside the network.}}

\newglossaryentry{CWND}{name=Congestion Window (CWND), description={The set of sequence numbers that a sender is allowed to have ``in flight'' (transmitted without being acknowledged yet).}}

\newglossaryentry{TCP}{name=Transmission Control Protocol (TCP), description={A protocol (actually: a family of protocol variants) that provides connection-oriented, dependable, in-order, congestion-controlled, flow-controlled byte streams (primarily) on top of a best-effort packet-switched network.}}

\newglossaryentry{UDP}{name=User Datagram Protocol (UDP), description={Simple transport protocol that basically only provides multiplexing service.}}

\newglossaryentry{SCTP}{name={Stream Control Transmission Protocol (SCTP)}, description={A transport protocol that combines datagram transmission with dependable delivery and congestion control, with additional features like support for multiple streams without head-of-line blocking between streams.}}

\newglossaryentry{QUIC}{name=QUIC, description={A transport protocol that combines ideas from TCP and SCTP with additional concepts for short delay on first packets. Often used in web browsing applications.}}

\newglossaryentry{fourtuple}{name=four tuple, description={The combination of source and destination IP and port numbers that can be used to identify a flow. Combined with the TCP protocol flag, it identifies a TCP connection.}}
