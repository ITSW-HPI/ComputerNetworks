\documentclass[tikz]{standalone}
\usepackage[utf8]{inputenc}
\usepackage[T1]{fontenc}
\usepackage{ifthen} 

\usepackage{../../templates/moeptikz}
\usepackage[utf8]{inputenc}
\usepackage[T1]{fontenc}

\usepackage{circledsteps}

\RequirePackage{xcolor}

%% HPI color definitions according to the design manual
% These do not exactly match the RGB values used in the Powerpoint slide master due to unknown reasons
\definecolor{hpiyellow}{RGB}{246,168,0}
\definecolor{hpiorange}{RGB}{221,97,8}
\definecolor{hpired}{RGB}{177,6,58}
\definecolor{hpigray}{RGB}{90,96,101}
\definecolor{hpiblue}{RGB}{0,122,158}


\renewcommand{\sfdefault}{neosans}
% Different font weights for neosans
\newcommand{\textl}[1]{{\fontseries{l}\selectfont #1}} % light
\newcommand{\textm}[1]{{\fontseries{m}\selectfont #1}} % medium, same as default weight
\newcommand{\textsb}[1]{{\fontseries{sb}\selectfont #1}} % semibold
\newcommand{\textmb}[1]{{\fontseries{mb}\selectfont #1}} % bold, same as \textbf
\newcommand{\texteb}[1]{{\fontseries{eb}\selectfont #1}} % extra bold
\newcommand{\textub}[1]{{\fontseries{ub}\selectfont #1}} % ultra bold

\tikzset{every picture/.style={/utils/exec={\sffamily}}}
\tikzset{flipflop RSflanke/.style={
  flipflop,
  flipflop def={t1=S, t2=C, c2=1, t3=R, t6=Q, t4={\ctikztextnot{Q}}}
}}


\tikzset{
  mechanicalSwitch/.pic={
    \coordinate (-inUp) at (135:2); 
    \coordinate (-inDown) at (235:2);
    \coordinate (-out) at (2,0);
    \coordinate (-center) at (0,0);
    
    \draw (0,0) circle [radius = 2cm];
    \draw [fill=gray!20] (0,0) circle [radius = 0.2cm];

    \draw (0, 0) -- (2, 0);
    \draw (135:.8) -- (135:2); 
    \draw (225:.8) -- (225:2); 

    \draw [fill=gray!20] (2, 0) circle [radius=0.05cm]; 
    \draw [fill=gray!20] (135:2) circle [radius=0.05cm]; 
    \draw [fill=gray!20] (225:2) circle [radius=0.05cm]; 

    
    \draw [thick] (0,0) -- (175:1.5); 

    \draw [dashed, <->, domain=135:225] plot ({cos(\x)}, {sin(\x)}); 
  },
  mechanicalSwitchClosed/.pic={
    \coordinate (-inUp) at (135:2); 
    \coordinate (-inDown) at (255:2);
    \coordinate (-out) at (2,0);
    \coordinate (-center) at (0,0);
    \draw (0,0) circle [radius = 2cm];
    \draw [fill=gray!20] (0,0) circle [radius = 0.2cm];

    \draw (0, 0) -- (2, 0);
    \draw (135:.8) -- (135:2); 
    \draw (225:.8) -- (225:2); 

    \draw [fill=gray!20] (2, 0) circle [radius=0.05cm]; 
    \draw [fill=gray!20] (135:2) circle [radius=0.05cm]; 
    \draw [fill=gray!20] (225:2) circle [radius=0.05cm]; 

    
    \draw [thick] (0,0) -- (135:2); 

    \draw [dashed, <->, domain=135:225] plot ({cos(\x)}, {sin(\x)}); 
  }
}


\usetikzlibrary{calc}
\usetikzlibrary{positioning}


\usetikzlibrary{positioning,decorations.pathreplacing, calc, shapes.multipart, ext.positioning-plus}



\begin{document}


\begin{tikzpicture}
  \label{fig:basics:msc:packet:multihop}

  \node [client, scale=0.5] at (0,0) (a) {A}; 
  \node [switch, scale=0.5] at (1,0) (sw1) {SW 1}; 
  \node [switch, scale=0.5] at (2,0) (sw2) {SW 2}; 
  \node [client, scale=0.5] at (4,0) (d) {D};

  \foreach \n in {a, sw1, sw2, d} \draw (\n) -- ++ (0, -3);

  % a to sw1: 1 latency, time per packet: 2
  % sw1 to sw2: 1 latency, time per packet: 1
  % sw2 to d: 2 latency, time per packet: 1
  \begin{scope}[y=0.2cm,yshift=-0.5cm]
  \draw [fill=hpiblue!20] (0, -1) -- ++ (1, -1) -- ++ (0, -2) -- ++ (-1, +1) -- ++ (0, +2) ; 

  \draw [fill=hpiblue!20] (1, -4.5) -- ++ (1, -1) -- ++ (0, -1) -- ++ (-1, +1) -- ++ (0, +1) ; 

  \draw [fill=hpiblue!20] (2, -7) -- ++ (2, -2) -- ++ (0, -1) -- ++ (-2, +2) -- ++ (0, +1) ; 
  
\end{scope}
\end{tikzpicture}

\begin{tikzpicture}
  \label{fig:basics:msc:circuit:multihop}

  \node [client, scale=0.5] at (0,0) (a) {A}; 
  \node [switch, scale=0.5] at (1,0) (sw1) {SW 1}; 
  \node [switch, scale=0.5] at (2,0) (sw2) {SW 2}; 
  \node [client, scale=0.5] at (4,0) (d) {D};

  \foreach \n in {a, sw1, sw2, d} \draw (\n) -- ++ (0, -8);


  \begin{scope}[y=0.2cm,yshift=-0.5cm]
  % establish circuit 
  \draw [hpiyellow, fill=hpiyellow!20] (0, -1) -- ++ (1, -1) -- ++ (0, -0.1) -- ++ (-1, +1) -- ++ (0, +0.1) ; 

  \draw [hpiyellow, fill=hpiyellow!20] (1, -2.2) -- ++ (1, -1) -- ++ (0, -0.1) -- ++ (-1, +1) -- ++ (0, +0.1) ; 

  \draw [hpiyellow, fill=hpiyellow!20] (2, -3.8) -- ++ (2, -2) -- ++ (0, -0.1) -- ++ (-2, +2) -- ++ (0, +0.1) ; 

  
  %confirm circuit setup 
  
  \draw [hpiyellow, fill=hpiyellow!40] (4, -6) -- ++ (-2, -2) -- ++ (0, -0.1) -- ++ (+2, +2)-- ++ (0, -0.1);
  
  \draw [hpiyellow, fill=hpiyellow!40] (2, -8.3) -- ++ (-1, -1) -- ++ (0, -0.1) -- ++ (+1, +1)-- ++ (0, -0.1);

  \draw [hpiyellow, fill=hpiyellow!40] (1, -9.6) -- ++ (-1, -1) -- ++ (0, -0.1) -- ++ (+1, +1)-- ++ (0, -0.1);


  \node [align=center] (wait) at (-1,-6) {Wait for \\ processing};
  \foreach \p in {(1, -2), (2, -3.6),  (2, -8.1),  (1, -9.5), (0, -10.9)} {
    \draw [->, hpired] (wait.east) -- (\p) ; 
  }
  \draw [hpiorange, decorate,decoration={brace,amplitude=6pt, raise=1pt}] (4,0) -- (4, -12) node [midway, right=10pt,anchor=west,align=left] {Circuit\\setup}; 

  
  % data

  \draw [fill=hpiblue!20]  (0, -11) -- ++ (+4, -4) -- ++ (0, -10)  -- ++(-4, +4) -- ++ (0, 10); 
   
  % destroy circuit
  \begin{scope}[yshift=-5cm]
  
  \draw [hpired, fill=hpired!20] (0, -1) -- ++ (1, -1) -- ++ (0, -0.1) -- ++ (-1, +1) -- ++ (0, +0.1) ; 

  \draw [hpired, fill=hpired!20] (1, -2.2) -- ++ (1, -1) -- ++ (0, -0.1) -- ++ (-1, +1) -- ++ (0, +0.1) ; 

  \draw [hpired, fill=hpired!20] (2, -3.8) -- ++ (2, -2) -- ++ (0, -0.1) -- ++ (-2, +2) -- ++ (0, +0.1) ; 

  %confirm circuit teardown
  
  \draw [hpired, fill=hpired!40] (4, -6) -- ++ (-2, -2) -- ++ (0, -0.1) -- ++ (+2, +2)-- ++ (0, -0.1);
  
  \draw [hpired, fill=hpired!40] (2, -8.3) -- ++ (-1, -1) -- ++ (0, -0.1) -- ++ (+1, +1)-- ++ (0, -0.1);

  \draw [hpired, fill=hpired!40] (1, -9.6) -- ++ (-1, -1) -- ++ (0, -0.1) -- ++ (+1, +1)-- ++ (0, -0.1);

    \draw [hpiorange, decorate,decoration={brace,amplitude=6pt, raise=1pt}] (4,-1) -- (4, -12) node [midway, right=10pt,anchor=west,align=left] {Circuit\\teardown}; 

\end{scope}


\end{scope}
\end{tikzpicture}



\end{document}